    %\documentclass[bigbox]{jarticle}
    \documentclass{jarticle}[11pt]
    %\documentstyle[bigbox,fancybox]{jarticle}
     
    % コマンドの定義
    %
    % コメントアウト用のコマンド
    %   複数行にまたがる記述をまとめてコメントアウトする際に利用できる
    %   \COMMENT{ .... } で .... の部分をコメントアウト
    \newcommand{\COMMENT}[1]{}
     
    % 以下は,表(srmmary.tex)で使用しているコマンド
    \newcommand{\lw}[1]{\smash{\lower2.ex\hbox{#1}}}
     
    % 図を参照するためのマクロ
    \newcommand{\figref}[1]{\makebox{図~\ref{#1}}}
     
    % 表を参照するためのマクロ
    \newcommand{\tabref}[1]{\makebox{表~\ref{#1}}}

    %% 使用しているパッケージ等があれば,宣言しておく
    \usepackage{ascmac}
    \usepackage{graphicx} 
    \usepackage{afterpage}
    % 以下のパラメータは,見易いように適宜調整する.
    \topmargin=-1cm
    \textheight=24cm
    \textwidth=15.5cm
    \oddsidemargin=-.2cm
    \evensidemargin=-.2cm
     
    \title{{\normalsize 情報工学実験C(ソフトウェア)報告書}\\
    (ネットワークプログラミング)\\
    } 

    \author{ 
      学生番号: 09429533 \\
      提出者: 高島和嗣
    }
     
    \date{
      提出日: 2020年 1月9日(木) \\ %% <-- 提出日を記載のこと
      締切日: 2020年 1月9日(木)
    }
     
    \begin{document}
    \maketitle
    %%%%%%%%%%%%%%%%%%%%%%%%%%%
    \section{クライアント・サーバモデルの通信の仕組み}
    クライアントは,通信相手のIPアドレスの取得後,ソケットを形成し,サーバと接続する.サーバに要求メッセージを送信し,サーバで処理された結果を受け取り処理を行う.以下にクライアントの処理の流れを記載する.
    \begin{enumerate}
    \item 通信相手のIPアドレスの取得 
    \item ソケットの形成
    \item 接続の確立
    \item 要求メッセージを送信
    \item 応答メッセージを受信
    \item 応答メッセージを処理
    \item ソケットの削除
    \end{enumerate}
    サーバは,ソケットを形成し,ソケットを名付け,クライアントからの要求メッセージを常に待ち,要求メッセージが到着したら,処理を行い,結果を送信する.以下にサーバの処理の流れを記載する.
    \begin{enumerate}
    \item ソケットの形成
    \item ソケットの名付け
    \item 接続要求の待機
    \item 接続要求の受理
    \item 要求メッセージを受信
    \item 要求メッセージを処理
    \item 応答メッセージを送信
    \end{enumerate}

%%%%%%%%%%%%%%%%%%%%%%%%%%%%%%%%%%%%%%%%%%%%%
    \section{名簿管理プログラムのクライアント・サーバプログラムの作成方針}
    今回の演習では,外部から入力されたデータを計算機で扱える内部形式に変換して格納し,それらの操作を行う名簿管理プログラムをクライアント・サーバ方式で作成した.
    名簿管理プログラムは,標準入力から「ID,氏名,生年月日,住所,備考」からなるCSV形式の名簿データを受け,それらをメモリに登録する機能と標準入力から\verb|%|で始まるコマンドを受け,登録してあるデータの件数を表示したり,データ自体を表示したり,整列したりする機能を持つ.
    名簿管理プログラムで処理を行うコマンド内容を以下に記載する.\\
    \begin{table}[htb]
      \label{tbl:コマンドリスト}
      \begin{tabular}{|l|l|l|}
        \hline
        コマンド & 解説 & パラメータ範囲 \\    
        \hline
        \verb|%Q| & プログラムを終了 & \\  
        \hline
        \verb|%C| & CSVデータの登録件数などを表示 & \\
        \hline
        \verb|%P n| & CSVデータの先頭から\verb|n|件表示 & \verb|n|: $0$ \verb|~| $99$ \\ & & ($n = 0,n > 100$:全件表示,\\ & &  $n < 0$:後ろから$-n$件表示)\\
        \hline
        \verb|%R file| & fileから読み込み & \\
        \hline
        \verb|%W file| & fileへ書き出し & \\
        \hline
        \verb|%F word| & wordを含むデータを検索 & \\
        \hline
        \verb|%S n| & データをn番目の項目で整列 & \\
        \hline
        \verb|%D n| & n番目のデータを削除 & \verb|n|: $0$ \verb|~| $99$ \\ & & ($n = 0$:全て削除,\\ && $n < 0,n > 100$:警告文の表示)\\
        \hline
        \verb|%V word| & wordを含むデータを削除 & \\
        \hline
        \verb|%H| & コマンド一覧 & \\
        \hline
        \verb|%B| & データ群を1つ前の状態に戻す & \\
        \hline
        \hline
\end{tabular}
\end{table}


クライアントで入力された内容をサーバに送信し,サーバ側で内容ごとに異なる処理を行うように,場合分けした.また,クライアント側での処理も必要であると思ったコマンドはクライアント側でも場合分けし,処理の方法を変えた.
具体的には,クライアント側で\verb|%Q|,\verb|%P|,\verb|%R|,\verb|%W|,\verb|%F|とそれ以外の場合とで分けた.
\verb|%Q|,\verb|%P|,\verb|%R|,\verb|%W|,\verb|%F|は,send,recvの数が複数個に及ぶ可能性があるため,それぞれの処理ごとに関数を作り,それ以外の場合は$1$つの\verb|send|に対して$1$つの\verb|recv|のみで処理出来るので同じ関数で処理する.
\verb|%Q|の処理時には,クライアント側のみ終了し,サーバ側は終了せずに次のクライアントからの接続要求の待機状態にするために,\verb|listen|関数で接続待機した後に\verb|while|文で$2$重ループし,$1$つ目のループで\verb|accept|関数で接続要求を受け入れ,$2$つ目のループでクライアントからのメッセージを受け取り,\verb|%Q|の時にはクライアントと接続したソケットを閉じ$2$つ目のループから抜け出すようにする.

%%%%%%%%%%%%%%%%%%%%%%%%%%%%%%%%%%%%%%%%
    \section{プログラムの説明}
\subsection{サーバー・クライアントプログラムの処理の流れ}
\begin{itemize}
\item クライアント 
  \begin{enumerate}  \item \verb|gethostbyname|関数を用いて通信先のIPアドレスを獲得する.
  \item \verb|socket|関数でソケットを作成し,\verb|connect|関数を用いるために\verb|sockaddr_in|構造体を\verb|sockaddr|構造体にキャストし,\verb|connect|関数でソケットを接続する.
  \item \verb|read|関数を用いて標準入力を行い,関数\verb|parse_line|で入力内容がコマンドか,データ入力かで場合分けする.
  \item コマンドなら,さらに\verb|%Q|,\verb|%P|,\verb|%R|,\verb|%W|,\verb|%F|とそれ以外の場合で処理を分ける.
  \end{enumerate}

\item サーバ
  \begin{enumerate}
  \item \verb|socket|関数でソケットを作成し,\verb|bind|関数でソケットに名前を付け,\verb|listen|関数でクライアント側からの接続要求を待つ.
  \item \verb|accept|関数でクライアントの接続要求を受け入れる.
  \item \verb|recv|関数でクライアントからのメッセージを受信する.この時,メッセージが\verb|%Q|ならば,\verb|close|関数でクライアントに接続されているファイルディスクリプタを閉じる.
  \item 関数\verb|parse_line|で入力内容がコマンドか,データ入力かで場合分けし,コマンドの種類でも場合分けし,処理を行う.
  \end{enumerate}
\end{itemize}

\subsection{サーバ,クライアントのプロトコル}
今回作成したプログラムのプロトコルはTCP/IP方式である.IPとは,インターネット上の機器が持つ識別番号であるIPアドレスに基づいて通信を行うプロトコルである.TCPとは,ポート番号という識別番号を用いて、IPアドレスの宛先のコンピュータのどのアプリケーションなのか識別できる。また,通信相手の状況を確認して接続を確立し,データの伝送が終わると切断する.相手がデータを受け取ったかを確認し,データの欠落や破損を検知した場合は再送したり,届いたデータを送信順に並べ直したりといった制御を行う.
通信する宛先のIPアドレスが分かれば,そのIPアドレス先にデータを送信できるが,どのアプリケーションでそれを受信するのか判断できない.そこで,TCPヘッダにポート番号を付加することで、どのアプリケーションと通信するか識別する.ポート番号は,$0\sim65535$の間で指定できるが$1023$番までのポートは,主要なプロトコルで用いられている. 

\subsection{サーバ,クライアントでのコマンドの処理内容}
クライアント側で,入力内容がコマンドの時に\verb|%|の後に空白がない時は空白を入れるようにエラー文を表示した.また,\verb|%Q|,\verb|%P|,\verb|%R|,\verb|%W|,\verb|%F|はそれぞれの関数で処理し,他のコマンドは関数\verb|cmd_default|で処理を行う.
サーバ側では,各コマンドごとの処理を行う前に引数の型が適切であるかチェックし,適切でない時はエラー文をクライアントに送信する.\\

\subsubsection{クライアント}
\begin{itemize}
\item \%Q \\
  関数\verb|cmd_quit|,でサーバに入力内容\verb|%Q|を送り,ソケットを閉じた後に,exit関数で処理を終了する.

\item \%P n \\
  関数\verb|cmd_print|で,サーバに入力内容\verb|%P n|を送り,応答メッセージを受け取る.この時,\verb|%P|の引数nが数字ではない時,応答メッセージの内容がmissとなっているので,エラー文を表示する.引数が適切の場合,応答メッセージは引数に応じた表示するデータの数なので,その数の分だけループ処理をし,データを受信し表示する.

\item \%R filename \\
関数\verb|cmd_read|で,ファイルを読み取り形式で開き,引数のファイル名がクライアントに存在しない時にはエラー文を表示する.存在する場合,ファイルの中身を$1$行ずつ読み取り,CSVとしてサーバに送信する.全行送り終えた後にファイルを閉じる.

\item \%W filename \\
関数\verb|cmd_write|で,ファイルを書き込み形式で開く.サーバに入力内容\verb|%W file|を送り,応答メッセージを受け取る.応答メッセージの内容はサーバに存在するデータの総数であるため,その数の分だけループ処理をし,受け取ったデータをファイルに書き込む.

\item \%F word \\
関数\verb|cmd_find|で,サーバに入力内容\verb|%F word|を送り,応答メッセージを受け取る.応答メッセージの内容はサーバ内の引数として与えた\verb|word|を含むデータの数であるため,$0$の時はクライアントから送られてきたエラー文を表示し,$1$以上の時はデータの数の分だけループ処理をし,受信したデータを表示する.

\item \%C ,\%S n ,\%D n ,\%V word ,\%H,\%B ,CSVデータ入力\\
  関数\verb|cmd_default|でサーバに入力内容を送り,サーバから応答メッセージを受け取り表示する.また,\verb|\%C|の時はサーバ内のデータの総数を記載したメッセージを受信し,\verb|\%H|の時は登録されているコマンド一覧を記載したメッセージを受信し,\verb|\%S,\%D,\%V,\%B|ではサーバ内のデータ群を変化させる処理なので,処理を完了したことを示すメッセージを受信した.CSVデータの入力の場合は,if文を用いてメッセージは表示しないようにした.

また,\verb|\%C|,\verb|\%H|は引数が必要ないので,引数が存在する時はエラー文を受信する.\verb|\%D|,\verb|\%B|の引数が数字でないときもエラー文を受信する.\%S,\%Vは引数を含むデータがサーバに存在しない時に,引数を含むデータが存在しないことを示す文を受信する.

\end{itemize}

\subsubsection{サーバ}
\begin{itemize}
\item \%Q \\
受信した内容が\verb|%Q|の時,\verb|main|部分でクライアントに接続されているファイルディスクリプタを閉じ,ループの外に出る.
また,この時サーバのソケットは閉じず,新しくクライアントと接続すると処理を再開する.

\item \%P n\\
コマンドPの引数nを$n>0,n=0,n<0$で場合分けを行い,$n=0$の時は全件表示,$n<0$の時は前から$n$件表示,$n<0$の時は後ろから$-n$件表示し,引数が登録されているデータ数より大きいときは全件表示する.
  また,表示したいデータを$1$件ずつクライアントに送信するため,クライアントとの\verb|send|,\verb|recv|の数を同数にするためにデータを送信する前に引数の絶対値を送信し,その分だけループ処理を行い,クライアントにデータを送信する.

\item \%R filename\\
クライアントからデータを$1$件ずつ送信されているので,サーバで\verb|%R|の処理を行われない.

\item \%W filename\\
サーバに格納されている全てのデータをクライアント側に送信する.また,サーバとクライアントとの\verb|send|,\verb|recv|の数を同数にするために予め,データの総数をクライアントに送信しておく.

\item \%F word\\
\verb|strcmp_word|関数で,引数の単語を含むデータが存在するか調べ,存在するデータの数をクライアントに送信する.データの数が$0$のときは,クライアントに引数を含むデータが存在しないことを示す文を送信し,$1$以上のときは,引数の単語を含むデータを全てクライアントに送信する.

\item \%C \\
サーバに登録されているデータの数を記載した文をクライアントに送信する.

\item \%S n\\
引数の値が$1\sim5$以外の時は,エラー文をクライアントに送信する.引数が$1\sim5$の時は,\verb|compare_profile|関数で入力された引数に応じて各メンバをソートする.比べる$2$つのデータが\verb|compare_profile|関数で設定した条件を満たすとき,\verb|swap|関数により順番を入れ替えるという操作を全データ分繰り返す.

\item \%D n\\
引数が$0$以下もしくは登録されているデータ数以上の時は,エラー文をクライアントに送信する.引数が$0$の時は,全件削除する.
引数$n$番目のデータを削除するために,$n$番目以降のデータを$1$つずつずらしていく処理を繰り返し,最後に登録されているデータの数を$1$少なくし,処理を完了したことを示すメッセージをクライアントに送信する.

\item \%V word\\
\verb|strcmp_word|関数で,引数の単語を含むデータが存在するか調べ,存在するデータの数をクライアントに送信する.データが存在しない時はクライアントに引数を含むデータが存在しないことを示すメッセージをクライアントに送信し,データが存在する時はそのデータ以降のデータを$1$つずつずらしていく処理を繰り返しデータを削除し,処理を完了したことを示すメッセージをクライアントに送信する.

\item \%H \\
実行可能なコマンドを全て記載した文をクライアントに送信する.

\item \%B \\
データ群の形を$1$つ前の状態に戻すために,実行後にデータ群の形が変わるコマンド(\verb|%R,%S,%D,%V|)のプログラムにそのコマンドの処理が起こる前に,$1$つ前のコマンドが処理した後のデータ群を別の構造体\verb|another[i]|に保存する.\%B実行時に構造体\verb|profile_data_store[i]|を\verb|another[i]|に置き換える.
\end{itemize}


%%%%%%%%%%%%%%%%%%%%%%%%%%%%%%%%%%%%%%%%%%%%%%%%%%%%%%%%%%%%%%%%%%%%%%%%%
    \section{プログラムの使用法}
先にサーバプログラムを実行し接続待機させておき,次にクライアントプログラムを実行しサーバと接続させた.

\begin{verbatim}
gcc directory_server.c -o server
./server &
gcc directory_client.c -o client
./client
\end{verbatim}

以下にコマンド実行時の処理の様子を記載する.
\begin{verbatim}
%H
%%Q      : プログラムを終了
%%C      : CSVデータの登録件数などを表示 
%%P n    : CSVデータの先頭からn件表示 (n: 0 - 99 ... n = 0,n > 100:全件表示,n < 0:後ろから-n件表示)
%%R file : fileから読み込み 
%%W file : fileへ書き出し 
%%F word : wordを含むデータを検索
%%S n    : データをn番目の項目で整列
%%D n    : n番目のデータを削除 (n: 0 - 99 ... n=0:全件削除,n < 0,n > 100:警告文の表示)
%%V word :  wordを含むデータを削除
%%B      :  データ群の形を1つ前の状態に戻す

%R sample2.txt
%C
11 profile(s)
%P 3
ID    : 5100046
Name  : The Bridge
Birth : 1845-11-02
Addr  : 14 Seafield Road Longman Inverness
Com.  : SEN Unit 2.0 Open

ID    : 5100127
Name  : Bower Primary School
Birth : 1908-01-19
Addr  : Bowermadden Bower Caithness
Com.  : 01955 641225 Primary 25 2.6 Open

ID    : 5100224
Name  : Canisbay Primary School
Birth : 1928-07-05
Addr  : Canisbay Wick
Com.  : 01955 611337 Primary 56 3.5 Open

%D 2
%D was executed
%P 3
ID    : 5100046
Name  : The Bridge
Birth : 1845-11-02
Addr  : 14 Seafield Road Longman Inverness
Com.  : SEN Unit 2.0 Open

ID    : 5100224
Name  : Canisbay Primary School
Birth : 1928-07-05
Addr  : Canisbay Wick
Com.  : 01955 611337 Primary 56 3.5 Open

ID    : 5100321
Name  : Castletown Primary School
Birth : 1913-11-04
Addr  : Castletown Thurso
Com.  : 01847 821256 01847 821256 Primary 137 8.5 Open

%B
%B was executed
%P 3
ID    : 5100046
Name  : The Bridge
Birth : 1845-11-02
Addr  : 14 Seafield Road Longman Inverness
Com.  : SEN Unit 2.0 Open

ID    : 5100127
Name  : Bower Primary School
Birth : 1908-01-19
Addr  : Bowermadden Bower Caithness
Com.  : 01955 641225 Primary 25 2.6 Open

ID    : 5100224
Name  : Canisbay Primary School
Birth : 1928-07-05
Addr  : Canisbay Wick
Com.  : 01955 611337 Primary 56 3.5 Open

%F The Bridge
ID    : 5100046
Name  : The Bridge
Birth : 1845-11-02
Addr  : 14 Seafield Road Longman Inverness
Com.  : SEN Unit 2.0 Open

%V The Bridge
%V was executed
%C
10 profile(s)
%S 2
%S was executed
%P 3
ID    : 5100127
Name  : Bower Primary School
Birth : 1908-01-19
Addr  : Bowermadden Bower Caithness
Com.  : 01955 641225 Primary 25 2.6 Open

ID    : 5100224
Name  : Canisbay Primary School
Birth : 1928-07-05
Addr  : Canisbay Wick
Com.  : 01955 611337 Primary 56 3.5 Open

ID    : 5100321
Name  : Castletown Primary School
Birth : 1913-11-04
Addr  : Castletown Thurso
Com.  : 01847 821256 01847 821256 Primary 137 8.5 Open

%Q
\end{verbatim}

%%%%%%%%%%%%%%%%%%%%%%%%%%%%%%%%%%%%%%%%%%%%%
    \section{プログラムの作成過程に関する考察}
    ファイルの読み込み及び書き込みのコマンドを実装する際,クライアント側のファイルを指定して処理を行うことにした.今回作成したクライアント,サーバのプログラムは同じパソコンで実装しているが,実際のクライアント・サーバという形式ではサーバとクライアントのコンピュータは別々で動作していることが多いと思い,その場合サーバ側に手を加えファイル操作することは難しいのではないかと思ったため,クライアント側でファイル操作することにした.

    クライアントの\verb|%P|,\verb|%W|,\verb|%F|はサーバからの複数個のデータを受け取るため,\verb|recv|関数が送られてくるデータ分必要である.しかし,クライアント側ではサーバから送信されるデータ吸うが分からないため,データを受け取る前に騒人されてくるデータ数を\verb|recv|しておき,その数だけサーバとクライアントでループを回し,データを受信した.

    また,作成途中の\verb|P|でエラーを検出した時にクライアント側の処理が停止し,入力を受け付けない時があった.この原因は,クライアントでコマンド\verb|P|が正常に動くときとエラーの時とで同じ\verb|recv|関数でメッセージを受信していたからである.エラーの時は$1$つのエラー文しか受信せず,\verb|%P|実行時は,複数のデータを受信するため数が正しくない.今回作成したサーバプログラムでは,\verb|exec_command|関数内で引数が適切であるかをチェックし,コマンドを実行するかエラー文を送信している.コマンドが実行された時にクライアントは最初にサーバのデータ数を受け取り,次にデータを受け取る処理だが,エラーの場合はエラー文を受け取った後に何も送信されないにもかかわらず,\verb|recv|関数がメッセージが送られてくるのを待機しているため,次の入力が出来なかった.これを解決するために\verb|if|文でエラーの時とで場合分けし,\verb|send|と\verb|recv|の数を調整した.

    コマンドを入力した時に処理結果が出力されず,次のコマンドを入力した時に$1$つ前のコマンドの処理結果が出力されることがあった.ソースコードを確認するとサーバとクライアントの\verb|send|と\verb|recv|の数が揃っておらず,正常に同期されていないことが分かった.サーバからクライアントの\verb|recv|の数よりも$1$つ多く\verb|send|していると,そのメッセージはクライアントの次の処理の\verb|recv|で受信されるため,処理結果が$1$つずれて表示されていたため,数を揃えることで解決した.


 \section{得られた結果に関する考察}
今回実装したプログラムの\verb|%R|はファイルの文を$1$行ずつサーバに送信するというものであるが,ファイルごとにサーバにアップロードするという形で実装すると,サーバ側でファイルを開くこともでき,別のクライアントからアップロードされたファイルをダウンロードすることもできると思った.クライアントで$1$行ずつ送ったものをサーバ側は書き込み形式で開いたファイルに書き込むことでファイルのアップロードができる.これを実装するには,新たにコマンドを作成するか,コマンドを拡張する必要があり,今回は時間が無かったため実装していない.

\section{感想}
以前作成した名簿管理プログラムをサーバとクライアントの形式で実装することで,ネットワークの通信の仕組みについて理解することができた.今回は時間が無く実装できなかったが,サーバとクライアントを1対多対応可能になるように拡張したいと思った.また,1からサーバとクライアント形式のプログラムを作成することがさらに深くネットワークを理解することにつながると思うので,何らかのサーバとクライアント形式のプログラムを作成したい.


%%%%%%%%%%%%%%%%%%%%%%%%%%%%%%%%%%%%%%%%%%%%%%
    \section{作成したプログラム}
    \subsection{サーバプログラム} \label{sec:サーバプログラム}
\begin{verbatim} 
     1	#include<stdio.h>
     2	#include<string.h>
     3	#include <stdlib.h>
     4	
     5	#include<fcntl.h>
     6	#include<netdb.h>
     7	#include <netinet/in.h>
     8	#include<sys/stat.h> 
     9	#include<sys/types.h> 
    10	#include<sys/socket.h>
    11	#include<arpa/inet.h>
    12	#include<errno.h>
    13	
    14	#define MAX_LINE_LEN 1024
    15	#define MAX_STR_LEN  69
    16	#define MAX_PROFILES 10000
    17	
    18	void parse_line(char *,int);
    19	
    20	 struct date {
    21	   int y;
    22	   int m;
    23	   int d;
    24	 };
    25	 
    26	 struct profile {
    27	   int         id;
    28	   char        name[MAX_STR_LEN+1];
    29	   struct date birthday;
    30	   char        home[MAX_STR_LEN+1];
    31	   char        *comment;
    32	 };
    33	
    34	struct profile profile_data_store[MAX_PROFILES];
    35	int profile_data_nitems=0; 
    36	int number;
    37	FILE *fp;
    38	struct profile another[MAX_PROFILES];
    39	
    40	//%%文字を別の文字に置換
    41	 int subst(char *str, char c1, char c2)
    42	 {
    43	   int n = 0;
    44	 
    45	   while (*str) {
    46	     if (*str == c1) {
    47	       *str = c2;
    48	       n++;
    49	    }
    50	     str++;
    51	   }
    52	   return n;
    53	 }
    54	
    55	//%%文を分割
    56	 int split(char *str, char *ret[], char sep, int max)
    57	 {
    58	   int n = 0;
    59	   ret[n++] = str; 
    60	   while (*str && n < max) {
    61	     if (*str == sep){
    62	      *str = '\0';
    63	      if(*(str+1) != sep){
    64	       ret[n++] = str + 1;
    65	      }
    66	     }
    67	     str++;
    68	   }
    69	  return n;
    70	 }
    71	
    72	struct date *new_date(struct date *d, char *str)
    73	  {
    74	   char *ptr[3];
    75	 
    76	   if (split(str, ptr, '-', 3) != 3){
    77	     return NULL;
    78	   } else{
    79	   d->y = atoi(ptr[0]);
    80	   d->m = atoi(ptr[1]);
    81	   d->d = atoi(ptr[2]);
    82	   return d;
    83	   }
    84	 }
    85	
    86	struct profile *new_profile(struct profile *p, char *csv,int fd)
    87	 {
    88	     char *qtr[5]; 
    89	     if (split(csv, qtr, ',', 5) != 5) {
    90	         return NULL; 
    91	     }
    92	     p->id = atoi(qtr[0]); 
    93	   
    94	     strncpy(p->name, qtr[1], MAX_STR_LEN); 
    95	     p->name[MAX_STR_LEN] = '\0';
    96	
    97	     if (new_date(&p->birthday, qtr[2]) == NULL) {
    98	         return NULL; 
    99	     }
   100	     strncpy(p->home, qtr[3], MAX_STR_LEN);
   101	     p->home[MAX_STR_LEN] = '\0';
   102	     
   103	     p->comment = (char *)malloc(sizeof(char) * (strlen(qtr[4])+1));
   104	     strcpy(p->comment, qtr[4]); 
   105	     return p;  
   106	 }
   107	
   108	////%%コマンド処理
   109	////コマンドC
   110	void cmd_check(fd)
   111	{
   112	    char buf[MAX_LINE_LEN];
   113	    sprintf(buf,"%d profile(s)\n",profile_data_nitems);
   114	    send(fd,buf,1024,0);
   115	}
   116	
   117	
   118	char *date_string(char buf[],struct date *date)
   119	{
   120	  sprintf(buf,"%04d-%02d-%02d",date->y,date->m,date->d);
   121	  return buf;
   122	}
   123	
   124	
   125	void print_profile(struct profile *p,int fd)
   126	{
   127	    char date[20];
   128	    char print_buf[1024];
   129	    int i;
   130	    sprintf(print_buf,"ID    : %d\n""Name  : %s\n""Birth : %s\n""Addr  : 
%s\n""Com.  : %s\n", p->id, p->name, date_string(date, &p->birthday), p->home, 
p->comment);
   131	    //for(i=0; print_buf[i]!='\0'; i++);
   132	    //printf("%s",print_buf);
   133	    send(fd,print_buf,1024,0); 
   134	}
   135	
   136	////コマンドP
   137	void cmd_print(int nitems, int fd)
   138	{
   139	    int i, start = 0,end = profile_data_nitems;
   140	    char num[1024];
   141	    char str[1024];
   142	
   143	    if (nitems > 0) {
   144	        if(nitems < profile_data_nitems) end = nitems;
   145	        sprintf(num,"%d",nitems);
   146	        send(fd,num,1024,0);
   147	    }
   148	    else if (nitems < 0) {
   149	        if(end + nitems > start) start = end + nitems ;
   150	        sprintf(num,"%d",-nitems);
   151	        send(fd,num,1024,0);      
   152	    } else {
   153	        sprintf(num,"%d",profile_data_nitems);   //全件表示
   154	        send(fd,num,1024,0); 
   155	    }
   156	
   157	    for (i = start; i < end; i++) {
   158	        recv(fd,str,1024,0);
   159	        print_profile(&profile_data_store[i],fd);
   160	    }
   161	}
   162	
   163	/*
   164	//コマンドR
   165	void cmd_read(char *filename,int fd)
   166	{
   167	  int l;
   168	  char line[MAX_LINE_LEN + 1];
   169	
   170	  if((fp = fopen(filename,"r")) == NULL){
   171	    fprintf(stderr, "%%R: file open error %s.\n", filename);
   172	    return;
   173	    }
   174	
   175	//////%D用///////
   176	   number=profile_data_nitems;
   177	  for(l=0; l<=number-1; l++){
   178	    another[l] = profile_data_store[l];
   179	  }
   180	  while(get_line(fp,line)){
   181	      parse_line(line,fd);
   182	  }
   183	  fclose(fp);
   184	}
   185	*/
   186	
   187	////コマンドC
   188	void fprint_profile_csv(int fd,struct profile *p)
   189	{
   190	  char buf[1024];
   191	  char date[20];
   192	  sprintf(buf,"%d,%s,%s,%s,%s",p->id,p->name,date_string(date,&p->birthday)
,p->home,p->comment);
   193	  send(fd,buf,1024,0);
   194	}
   195	
   196	/////コマンドW
   197	void cmd_write(char *filename,int fd)
   198	{
   199	  int i,l;
   200	  char num[10],str[10];
   201	  if((fp = fopen(filename,"w")) == NULL){
   202	      fprintf(stderr,"file error");
   203	    }
   204	  sprintf(num,"%d",profile_data_nitems);
   205	  send(fd,num,10,0); 
   206	
   207	  number=profile_data_nitems;
   208	  for(i=0; i<=number-1; i++){
   209	    another[i] = profile_data_store[i];
   210	  }
   211	  
   212	  for (i = 0; i < profile_data_nitems; i++) {
   213	      recv(fd,str,10,0);
   214	      fprint_profile_csv(fd,&profile_data_store[i]);
   215	  }
   216	  fclose(fp);
   217	}
   218	
   219	
   220	int strcmp_word(struct profile *p,char *word)
   221	{
   222	  char id[20];
   223	  char date[20];
   224	 sprintf(id,"%d",p->id);
   225	 if(strcmp(id, word) == 0||strcmp(p->name, word) == 0||strcmp(date_string(date, 
&p->birthday), word) == 0||strcmp(p->home, word) == 0||strcmp(p->comment, word) == 0){
   226	   return 0;
   227	 } 
   228	}
   229	
   230	////コマンドF
   231	void cmd_find(char *word,int fd)
   232	{
   233	    int i,times=0;
   234	    struct profile *p,*q;
   235	    char num[10],buf[1024];
   236	    char str[10];
   237	
   238	    memset(buf,0,MAX_LINE_LEN);
   239	
   240	    for (i=0; i < profile_data_nitems; i++) {
   241	        p=&profile_data_store[i];
   242	        if(strcmp_word(p,word)==0){ 
   243	            times++;
   244	        }
   245	    }
   246	    sprintf(num,"%d",times);
   247	    send(fd,num,10,0);
   248	    if(times>0){
   249	        for (i=0; i < profile_data_nitems; i++) {
   250	            q=&profile_data_store[i];
   251	            if(strcmp_word(q,word)==0){
   252	                recv(fd,str,10,0);
   253	                print_profile(q,fd);
   254	            }
   255	        }
   256	    }else{                    //times==0 : %F の引数を含むデータがないとき
   257	        recv(fd,str,10,0);
   258	        sprintf(buf,"no data with that phrase!\n");
   259	        send(fd,buf,1024,0); 
   260	    }
   261	}
   262	
   263	
   264	void swap(struct profile *p1, struct profile *p2)
   265	{
   266	  struct profile tmp;
   267	
   268	  tmp = *p1;
   269	  *p1 = *p2;
   270	  *p2 = tmp;
   271	}
   272	
   273	int compare_date(struct date *d1, struct date *d2)
   274	{
   275	  if (d1->y != d2->y) return (d1->y) - (d2->y);
   276	  if (d1->m != d2->m) return (d1->m) - (d2->m);
   277	  return (d1->d) - (d2->d);
   278	}
   279	
   280	//引数でソートを場合分け
   281	int compare_profile(struct profile *p1, struct profile *p2, int column)
   282	{
   283	  switch (column) {
   284	    case 1:
   285	      return p1->id - p2->id; break; 
   286	    case 2:
   287	      return strcmp(p1->name,p2->name); break;  
   288	    case 3:
   289	      return compare_date(&p1->birthday,&p2->birthday); break;  
   290	    case 4:
   291	      return strcmp(p1->home,p2->home); break;  
   292	    case 5:
   293	      return strcmp(p1->comment,p2->comment); break; 
   294	    }
   295	}
   296	
   297	////コマンドS
   298	void cmd_sort(int column,int fd)                   //エラー処理不十分
   299	{
   300	  int length =profile_data_nitems;
   301	  int i,j,l,s;
   302	  struct profile *p;
   303	  char buf[1024],num[10];
   304	  memset(buf,0,MAX_LINE_LEN);
   305	  s = length-1;
   306	
   307	  if(column != 1 && column != 2 && column != 3 && column != 4 && column !=5){
   308	      sprintf(buf,"%d is not adaptted\n",column);
   309	      send(fd,buf,1024,0);
   310	  } 
   311	  if(column == 1 || column == 2 || column == 3 || column == 4 || column ==5){
   312	      number=profile_data_nitems;
   313	      for(l=0; l<=number-1; l++){
   314	          another[l] = profile_data_store[l];           //コマンドD用
   315	      }
   316	      for(i = 0; i <= s; i++) {
   317	          for (j = 0; j <= s - 1; j++) {
   318	              p=&profile_data_store[j];
   319	              if (compare_profile(p, (p+1), column) > 0)
   320	                  swap(p, (p+1));  
   321	          }
   322	      }
   323	      sprintf(buf,"%%S was executed\n");
   324	      send(fd,buf,1024,0);
   325	  }
   326	}
   327	
   328	////コマンドD
   329	void cmd_delete(int nitems,int fd)
   330	{
   331	  int i, l, end = profile_data_nitems-1;
   332	  char buf[1024];
   333	  memset(buf,1024,0);
   334	  if(nitems < 0){
   335	      sprintf(buf,"%d is smaller than 0\n",nitems);
   336	      send(fd,buf,1024,0); 
   337	  } 
   338	  else if(nitems > end+1){
   339	      sprintf(buf,"%d is bigger than data number\n",nitems);
   340	      send(fd,buf,1024,0); 
   341	  }
   342	  else{
   343	      number=profile_data_nitems;
   344	      for(l=0; l<=number-1; l++){               
   345	          another[l] = profile_data_store[l];         //コマンドD用
   346	      }
   347	      if(nitems > 0 && nitems < end+1){
   348	          for(i=nitems-1;i<end;i++){
   349	              profile_data_store[i]=profile_data_store[i+1];  
   350	          }
   351	          profile_data_nitems--;
   352	      }
   353	      else if(nitems == end+1){
   354	          profile_data_nitems--;
   355	      }
   356	      else if(nitems == 0){         //ALL DELETE
   357	          profile_data_nitems=0;
   358	      }
   359	      sprintf(buf,"%%D was executed\n");
   360	      send(fd,buf,1024,0);
   361	  }
   362	}
   363	
   364	////コマンドV
   365	void cmd_vanish(char *word,int fd)
   366	{
   367	    int i,k,l,times=0;
   368	    char date[20];
   369	    char id[20];
   370	    char buf[1024];
   371	    memset(buf,1024,0);
   372	    struct profile *p;
   373	    for (i=0; i < profile_data_nitems; i++) {
   374	        p=&profile_data_store[i];
   375	        if(strcmp_word(p,word)==0){
   376	            times++;
   377	        }
   378	    }
   379	    if(times>0){
   380	        number=profile_data_nitems;
   381	        for(l=0; l<=number-1; l++){
   382	            another[l] = profile_data_store[l];             //コマンドD用
   383	        }
   384	        for (i=0; i < profile_data_nitems; i++) {
   385	            p=&profile_data_store[i];
   386	            if(strcmp_word(p,word)==0){
   387	                for(k=i;k<profile_data_nitems-1;k++){
   388	                    profile_data_store[k]=profile_data_store[k+1];  
   389	                }
   390	                profile_data_nitems--;
   391	            }
   392	        }
   393	        sprintf(buf,"%%V was executed\n");
   394	        send(fd,buf,1024,0);
   395	    } else {
   396	        sprintf(buf,"no data with that phrase!\n");
   397	        send(fd,buf,1024,0);
   398	    }
   399	}
   400	
   401	
   402	void cmd_help(int fd)
   403	{
   404	    char help_buf[1024]={"\0"};
   405	    int i;
   406	    char *str1  = "%%Q      : プログラムを終了\n";   
   407	    char *str2  = "%%C      : CSVデータの登録件数などを表示 \n";
   408	    char *str3  = "%%P n    : CSVデータの先頭からn件表示 (n : 0-99 ... n=0,
n>100 : 全件表示,n<0:後ろから-n件表示)\n";
   409	    char *str4  = "%%R file : fileから読み込み \n";
   410	    char *str5  = "%%W file : fileへ書き出し \n";
   411	    char *str6  = "%%F word : wordを含むデータを検索\n";     
   412	    char *str7  = "%%S n    : データをn番目の項目で整列\n";   
   413	    char *str8  = "%%D n    : n番目のデータを削除 (n: 0 - 99 ... n=0:全件削除,n < 0,n > 100:警告文の表示)\n"; 
   414	    char *str9  = "%%V word :  wordを含むデータを削除\n";
   415	    char *str10 ="%%B      :  データ群の形を1つ前の状態に戻す\n";
   416	    sprintf(help_buf,"%s%s%s%s%s%s%s%s%s%s\n",str1,str2,str3,str4,str5,str6,
str7,str8,str9,str10);
   417	   
   418	    for(i=0; help_buf[i]!='\0'; i++);
   419	    send(fd,help_buf,i,0);
   420	}
   421	
   422	
   423	void cmd_back(int fd){
   424	    char buf[1024];
   425	  int i,s=profile_data_nitems;
   426	  profile_data_nitems=number;
   427	  for(i=0; i<=profile_data_nitems-1; i++){
   428	    profile_data_store[i] = another[i];
   429	  }
   430	        sprintf(buf,"%B was executed\n");
   431	        send(fd,buf,1024,0);
   432	}
   433	
   434	
   435	int check1(char *param){
   436	if((*param>='a'&& *param<='z') || (*param>='A' && *param<='Z') 
|| (*param>='0' && *param<='9')) {
   437	  return 1;
   438	 }
   439	}
   440	
   441	int check2(char *param){
   442	if(*param>='0' && *param<='9') {
   443	  return 1;
   444	 }
   445	}
   446	
   447	int check3(char *param){
   448	    int l,i,j=0;
   449	    for(l=0;param[l]!='\0';l++);
   450	
   451	    for(i=0;i<l;i++){
   452	        if((param[i]>='a'&& param[i]<='z')||(param[i]>='A' && param[i]<='Z')){
   453	            j++;
   454	        }
   455	    }
   456	    return j;
   457	}
   458	
   459	/*
   460	int check3(char *param){
   461	  if((*param>='a'&& *param<='z') || (*param>='A' && *param<='Z')) {
   462	    return 1;
   463	  }
   464	}
   465	*/
   466	void exec_command(char cmd, char *param, int fd)
   467	{
   468	    char buf[1024];
   469	    memset(buf,0,MAX_LINE_LEN);
   470	    switch (cmd){
   471	    case 'C': 
   472	        if(check1(param)==1) {
   473	            sprintf(buf,"Don't write word\n");
   474	            send(fd,buf,1024,0);
   475	            break;
   476	        }else{ 
   477	            cmd_check(fd); 
   478	            break;
   479	        }
   480	    case 'P': 
   481	        if(check3(param)!=0) {
   482	            sprintf(buf,"miss");
   483	            send(fd,buf,1024,0);
   484	            break;
   485	        }else{
   486	            cmd_print(atoi(param),fd);
   487	            break;
   488	        }
   489	    case 'W': cmd_write(param,fd);       break;
   490	    case 'F': cmd_find(param,fd);     break;
   491	    case 'S': 
   492	        if(check3(param)!=0) {
   493	            sprintf(buf,"Please write number\n");
   494	            send(fd,buf,1024,0);
   495	            break;
   496	        }else{
   497	        cmd_sort(atoi(param),fd);  break;
   498	        }
   499	    case 'D': 
   500	        if(check3(param)!=0) {
   501	            sprintf(buf,"Please write number\n");
   502	            send(fd,buf,1024,0);
   503	            break;
   504	        }else{
   505	            cmd_delete(atoi(param),fd);
   506	            break;
   507	        }
   508	    case 'V': cmd_vanish(param,fd);     break;
   509	        
   510	    case 'H': 
   511	        if(check1(param)==1) {
   512	            sprintf(buf,"Don't write word\n");
   513	            send(fd,buf,1024,0);
   514	            break;
   515	        }else{ 
   516	            cmd_help(fd); 
   517	            break;
   518	        }
   519	    case 'B':
   520	        if(check1(param)==1) {
   521	            sprintf(buf,"Don't write word\n");
   522	            send(fd,buf,1024,0);
   523	            break;
   524	        }else{ 
   525	            cmd_back(fd); 
   526	            break;
   527	        }
   528	    default:
   529	        sprintf(buf,"command %c is ignored.\n",cmd);
   530	        send(fd,buf,1024,0);
   531	        break;
   532	    }
   533	}
   534	
   535	////%%parse_line
   536	void parse_line(char *line,int fd)
   537	{
   538	    int x=0;
   539	    char buf[3];
   540	    memset(buf,0,3);
   541	    if(line[0] == '%'){
   542	        exec_command(line[1],&line[3],fd);
   543	    }
   544	    else{
   545	/*
   546	        if(new_profile(&profile_data_store[profile_data_nitems],line,fd)
==NULL){
   547	            send(fd,"error",10,0);
   548	        } else{
   549	*/
   550	        new_profile(&profile_data_store[profile_data_nitems], line,fd);
   551	        send(fd,"ok",3,0);
   552	        profile_data_nitems++;
   553	        //}
   554	    }
   555	}
   556	
   557	int main (){
   558	    struct sockaddr_in read_sa;
   559	    struct sockaddr_in write_sa;
   560	    struct hostent *host;
   561	    int sockfd,new_sockfd;
   562	    int buf_len,write_len,i;
   563	    char recv_buf[1024],send_buf[1024];
   564	    int x=1;
   565	    memset(recv_buf,0,MAX_LINE_LEN);
   566	    memset(send_buf,0,MAX_LINE_LEN);
   567	
   568	/*make socket*/
   569	    if ((sockfd = socket(AF_INET, SOCK_STREAM, 0)) < 0){
   570	            fprintf(stderr,"socket error\n");
   571	            return 1;
   572	        }
   573	
   574	/*socket setting*/
   575	    memset((char*)&read_sa,0,sizeof(read_sa));
   576	    read_sa.sin_family      = AF_INET; // host address type
   577	    read_sa.sin_port        = htons(3001); // port number
   578	    read_sa.sin_addr.s_addr = htonl(INADDR_ANY);
   579	
   580	    if (bind(sockfd, (struct sockaddr *)&read_sa, sizeof(read_sa)) < 0){
   581	        fprintf(stderr,"bind error\n");
   582	            return 1;
   583	    }
   584	    
   585	    if(listen(sockfd,5) < 0 ){
   586	        fprintf(stderr,"listen error\n");
   587	        close(sockfd);
   588	        return 1;
   589	    }
   590	   
   591	     while(1){
   592	        if((new_sockfd = accept(sockfd,(struct sockaddr*)&write_sa,&write_len)) 
== -1) {
   593	            fprintf(stderr,"accept error\n");
   594	            return 1;
   595	        }
   596	        while(1){
   597	            recv(new_sockfd,recv_buf,1024,0);
   598	
   599	            //コマンドQの処理
   600	            if(recv_buf[0]=='%' && recv_buf[1]=='Q'){
   601	                sprintf(send_buf,"exit!");
   602	                send(new_sockfd,send_buf,1024,0);
   603	                close(new_sockfd);
   604	                break;
   605	            }
   606	
   607	            parse_line(recv_buf,new_sockfd);  
   608	        }
   609	     }
   610	}
\end{verbatim}

\subsubsection{クライアントプログラム} \label{sec:クライアントプログラム}
\begin{verbatim} 
     1	#include<stdio.h>
     2	#include<string.h>
     3	#include <stdlib.h>
     4	
     5	#include<netdb.h>
     6	#include<sys/types.h> /* 「注意」参照 */
     7	#include<sys/socket.h>
     8	#include<arpa/inet.h>
     9	#include <fcntl.h>
    10	#include<unistd.h>
    11	
    12	#define MAX_LINE_LEN 1024
    13	FILE *fp;
    14	struct hostent *hp;
    15	struct sockaddr_in sa;
    16	
    17	int sockfd;
    18	void parse_line(char *);
    19	
    20	int check1(char *param){
    21	    if((*param>='a'&& *param<='z') || (*param>='A' && *param<='Z') || 
    22	       (*param>='0' && *param<='9')) {
    23	        return 1;
    24	    }
    25	}
    26	
    27	 int subst(char *str, char c1, char c2)
    28	 {
    29	   int n = 0;
    30	 
    31	   while (*str) {
    32	     if (*str == c1) {
    33	       *str = c2;
    34	       n++;
    35	    }
    36	     str++;
    37	   }
    38	   return n;
    39	 }
    40	
    41	 int split(char *str, char *ret[], char sep, int max)
    42	 {
    43	   int n = 0;
    44	   ret[n++] = str; 
    45	   while (*str && n < max) {
    46	     if (*str == sep){
    47	      *str = '\0';
    48	      if(*(str+1) != sep){
    49	       ret[n++] = str + 1;
    50	      }
    51	     }
    52	     str++;
    53	   }
    54	  return n;
    55	 }
    56	
    57	int get_line(FILE *fp,char *line)
    58	 {
    59	   if (fgets(line, MAX_LINE_LEN + 1, fp) == NULL){
    60	     return 0;
    61	 } else{
    62	   subst(line, '\n', '\0');
    63	   return 1;
    64	   }
    65	 }
    66	
    67	//コマンドQ
    68	void cmd_quit(char *line)
    69	{
    70	    char buf[1024];
    71	    send(sockfd,line,MAX_LINE_LEN,0);
    72	    recv(sockfd,buf,MAX_LINE_LEN,0);
    73	    close(sockfd);
    74	    exit(0);
    75	}
    76	
    77	//コマンドP
    78	void cmd_print(char *line){
    79	    int i,nitems;
    80	    char num[1024],buf[MAX_LINE_LEN];
    81	    memset(buf,0,MAX_LINE_LEN);
    82	    memset(num,0,1024);
    83	
    84	    send(sockfd,line,MAX_LINE_LEN,0);
    85	    recv(sockfd,num,1024,0);
    86	
    87	    if(strcmp(num,"miss")==0){
    88	        printf("Please write number\n");
    89	    }else{
    90	        nitems=atoi(num);
    91	        
    92	        for(i=0;i<nitems;i++){
    93	            send(sockfd,"do",10,0);
    94	            recv(sockfd,buf,MAX_LINE_LEN,0);
    95	            printf("%s",buf);
    96	        }
    97	    }
    98	}
    99	
   100	//コマンドR
   101	void cmd_read(char *filename){
   102	    char read_buf[MAX_LINE_LEN],recv_buf[MAX_LINE_LEN];
   103	    int i,buf_len,x=1;
   104	
   105	    if((fp = fopen(filename,"r")) == NULL) {
   106	        fprintf(stderr, "%%R: file open error %s.\n", filename);
   107	        return;
   108	    }
   109	
   110	    memset(read_buf,0,MAX_LINE_LEN);
   111	    while(fgets(read_buf,MAX_LINE_LEN,fp) != NULL){
   112	        //printf("%s",read_buf);  //確認用
   113	        send(sockfd,read_buf,MAX_LINE_LEN,0);
   114	        recv(sockfd,recv_buf,MAX_LINE_LEN,0);
   115	        memset(read_buf,0,MAX_LINE_LEN);
   116	    }
   117	   
   118	    fclose(fp);
   119	}
   120	
   121	//コマンドW
   122	void cmd_write(char *line,char *filename){
   123	    int i,nitems;
   124	    char num[10],buf[1024];
   125	
   126	     if((fp = fopen(filename,"w")) == NULL){
   127	      fprintf(stderr,"error");
   128	      return ;
   129	    }
   130	
   131	     memset(buf,0,MAX_LINE_LEN);
   132	     memset(num,0,10);
   133	
   134	     send(sockfd,line,MAX_LINE_LEN,0);
   135	     recv(sockfd,num,10,0);
   136	     nitems=atoi(num);
   137	
   138	     for (i = 0; i < nitems; i++) {
   139	         send(sockfd,"do",10,0);
   140	         recv(sockfd,buf,MAX_LINE_LEN,0);
   141	         fprintf(fp,"%s",buf);
   142	  }
   143	     fclose(fp);
   144	}
   145	
   146	//コマンドF
   147	void cmd_find(char *line,char *param){
   148	    char recv_buf[MAX_LINE_LEN],num[10];
   149	    int i,k,times;
   150	    memset(recv_buf,0,MAX_LINE_LEN);
   151	
   152	    for(i=0; line[i]!= '\0'; i++);
   153	    send(sockfd,line,i,0);
   154	    recv(sockfd,num,10,0);
   155	    times=atoi(num);
   156	    if(times==0) {
   157	        send(sockfd,"miss",10,0);
   158	        recv(sockfd,recv_buf,1024,0);
   159	        printf("%s",recv_buf);
   160	    }
   161	    for(i=0;i<times;i++){
   162	        send(sockfd,"do",10,0);
   163	        recv(sockfd,recv_buf,1024,0);
   164	        printf("%s",recv_buf);
   165	    }
   166	}
   167	
   168	//コマンド デフォルト(%S,%D,%V,%H)
   169	void cmd_default(char *line){
   170	    char recv_buf[MAX_LINE_LEN];
   171	    int buf_len,i;
   172	    memset(recv_buf,0,MAX_LINE_LEN);
   173	    //for(i=0; line[i]!= '\0'; i++);
   174	    send(sockfd,line,1024,0);
   175	    recv(sockfd,recv_buf,1024,0);
   176	    if(line[0]=='%'){
   177	        printf("%s",recv_buf);
   178	    }
   179	}
   180	
   181	void exec_command(char cmd, char *param,char *line){
   182	    int i,buf_len;
   183	    char buf[1024];
   184	
   185	    switch (cmd){
   186	    case 'Q': cmd_quit(line);                   break;
   187	    case 'P': cmd_print(line);              break;
   188	    case 'R': cmd_read(param);              break;
   189	    case 'W': cmd_write(line,param);        break;
   190	    case 'F': cmd_find(line,param);         break;
   191	    default:
   192	        cmd_default(line);                  break;
   193	    }
   194	}
   195	
   196	void parse_line(char *line)
   197	{
   198	    int i,buf_len;
   199	    char buf[MAX_LINE_LEN];
   200	    if(line[0] == '%'){
   201	        if(check1(&line[2])==1) {
   202	            printf("Please space after %c%c\n",line[0],line[1]);
   203	        }
   204	        else {
   205	            exec_command(line[1],&line[3],line);
   206	        }
   207	    }
   208	    else{
   209	        cmd_default(line);
   210	    }
   211	}
   212	
   213	void send_input(){
   214	    char buf[MAX_LINE_LEN];
   215	
   216	    int i;
   217	    memset(buf,0,MAX_LINE_LEN);
   218	    read(0,buf,MAX_LINE_LEN);
   219	    //printf("%s",buf);      //入力確認
   220	    subst(buf, '\n', '\0');
   221	    parse_line(buf);
   222	}
   223	
   224	int make_sockfd(){
   225	    int fd;
   226	    if ((fd = socket(AF_INET, SOCK_STREAM, 0)) < 0){
   227	        fprintf(stderr,"socket error\n");
   228	        return -1;
   229	    }
   230	
   231	    //memset(&sa,0,sizeof(sa));
   232	    sa.sin_family      = AF_INET; // host address type
   233	    sa.sin_port        = htons(3001); // port number
   234	    bzero((char*)&sa.sin_addr,sizeof(sa.sin_addr));
   235	    memcpy((char*)&sa.sin_addr,(char*)hp->h_addr,hp->h_length);
   236	    
   237	    if(connect(fd, (struct sockaddr*)&sa, sizeof(sa)) < 0){
   238	        fprintf(stderr,"connect error\n");
   239	        return -1;
   240	    }
   241	    return fd; 
   242	}
   243	
   244	int main (){
   245	    hp = gethostbyname("localhost");
   246	
   247	    if (hp==NULL){
   248	        fprintf(stderr,"ホスト取得失敗\n");
   249	        return 1;
   250	    }
   251	
   252	    if((sockfd=make_sockfd()) < 0) return;
   253	           
   254	    while(1){
   255	        send_input();
   256	    }
   257	}
   258	
\end{verbatim}

\end{document}
